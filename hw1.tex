%%%%%%%%%%%%%%%%%%%%%%%%%%%%%%%%%%%%%%%%%%%%%%%%%%%%%%%%%%%%%%%%%%%%%%%%%%%%%%%%%%%%
%%%%%%%%%%%%%%%%%%%free online editor available at%%%%%%%%%%%%%%%%%%%%%%
%%%%%https://www.writelatex.com/%%%%%%%%%%%%%%%%%%%%%%%%%%%%%%%%%%%%%%%%%%

\documentclass[10pt,leqno ]{article}
    %The document class defines the master templates, the structure of the document, 
    %and lays out the types of 
    %objects that can be manipulated for this type of document. 
    %the brackets contain basic options that will be applied globally (throughout
    %the document). Here, we specify a 10pt font, and when we number an equation, the 
    %number will be on the left.
    %The document class is a file ***.cls. You will probably never have to edit or create
    % a .cls file. There are many available on the internet for your use.

%%%%%%%%%%%%%%%%%%%%%%%%%%%%%%%%%%%%%%%%%%%%%%%%%%%%%%%%%%%%%%%%%%%%%%%%%%%%%%%%%%%%%%%%%%%%%%%%%%%%%%%%%%%%%%%%%%%%%%%%%%%%%%%%%%%%%%%%%%%%%%%%%%%%%%%%%%%%%%%%%%%%%%%%%%%%%%%%%%%%%%%%%%%%%%%%%%%%%%%%%%%%%%%%%%%%%%%%%%%%%%%%%%%%%%%%%%%%%%%%%%%%%%%%%%%%
\usepackage{amsfonts}
\usepackage[shortlabels]{enumitem}
\usepackage{amssymb}
\usepackage{amsmath}
\usepackage{times}
\usepackage{amsthm}
\usepackage{hyperref}
%\usepackage{homework}
    %packages control the ``style'' or look of the document. These come in the form of 
    %files ***.sty. The package ``homework'' above was created by me. The other packages
    %are very common for this type of document. You can google to learn more about what
    %they can do, and what options they give you. For example
\usepackage{textcomp}
\usepackage[margin=1.5in]{geometry}
    %the geometry package lets you customize the margins of your document.
    % and the 

\usepackage{forest}
  % For drawing the tree illustrating the sample space  
\usepackage{setspace}
    %package gives us the ability to set the line spacing.
\usepackage{moreenum}   % for greek letters in enum

\newtheorem{theorem}{Theorem}
\theoremstyle{definition} 
\newtheorem{problem}[theorem]{Problem}
    %these set up environments for listing things. The numbering is automatic.

    
\newenvironment{solution}[1][Solution]{\begin{doublespace}\textbf{#1.}\quad }{\ \rule{0.5em}{0.5em}\end{doublespace}}
    %this is the environment for writing solutions. Doble spaced, with an end of proof
    %box at the end
    
\title{Ordinary Differential Equations\\
Math 501, Fall 2019\\
Homework 1}
\author{Guy Matz \\
BMCC}
    %above is the information that goes in the title. Notice the { and }. 
    %the double slashes \\ mean start a new line.


\begin{document} %this means end the preamble (stuff controling the styles above and
%start the content of the document. We can make adjustments as we go. For example,
\maketitle
\begin{enumerate}
\item Number of bacteria after 90 min = 6049.65
\item Minutes until the salt concentration is 6 g/L = 367 min
\end{enumerate}
 
\newpage
\vskip .25in %skip a bit before we start the regular text.
\thispagestyle{empty} %no need to number first page.

\begin{problem}  \textbf{Exponential Growth} \\
Suppose a small colony of 1,000 bacteria, at t = 0 minutes is in a large bottle of water with lots of food and no predators (so there are no predation or density dependent issues to worry about). Suppose after 45 minutes there are 2,500 bacteria in the colony. Let y(t) = the number of bacteria in the colony at time t, t measured in minutes.

\begin{enumerate}
\item Write the ODE IVP for the above scenario
\begin{equation*}
\frac{dy}{dt} = ky, \, y(0) = 1,000
\end{equation*}
\item and find the solution to the IVP.
\begin{align*}
    \frac{dy}{dt} &= ky \\
    \frac{dy}{y} &= k\, dt \\
    \int \frac{dy}{dt} &= \int ky \\
    ln |y| &= kt + c_1 \\
    |y| &= e^{kt}e^{c_1} \\
    y &= Ce^{kt}
\end{align*}
With $y(0) = 1,000$,
\begin{align*}
y(0) = Ce^{k \cdot 0} = C \\
\end{align*}
and
\begin{align*}
y(0) &= C = 1,000
\end{align*}
so
\begin{align*}
y(t) = 1,000 e^{kt}
\end{align*}
\item Use the solution to estimate how many bacteria were in the bottle at t = 90 minutes.
\\
We first find $k$ at $t = 45$,
\begin{align*}
y(t) &= 1,000e^{kt} \\
\frac{y(t)}{1,000} &= e^{kt} \\
\text{ln } \frac{y(t)}{1,000} &= kt \\
\frac{\text{ln } \frac{y(t)}{1,000}}{t} &= k \\
\frac{\text{ln } \frac{y(45)}{1,000}}{45} &= k \\
\frac{\text{ln } \frac{2,500}{1,000}}{45} &= k \\
\frac{\text{ln } 2.5}{45} &= k \\
\frac{0.91629}{45} &= k \\
0.02 &= k
\end{align*}
Hence,
\begin{equation*}
y(t) = 1,000e^{0.02t}
\end{equation*}
And so at $t = 90$,
\begin{align*}
y(90) &= 1,000 e^{0.02 \cdot 90} \\
         &= 6049.65
\end{align*}
\end{enumerate}


\end{problem}

\begin{problem} \textbf{Mixing} \\
We start with 2,000 liters of seawater in a tank. In seawater there are 35 g of salt per liter. Suppose we continually pour fresh water (0.5 g salt/liter) into the top of the tank at the rate of 10 liters/minute and at the bottom of the tank, we continually drain off 10 liters/minute. So the amount of water in the tank is always 2,000 liters. Assume that the water in the tank is being stirred so that the saltwater and the freshwater mix immediately. How long until the concentration of salt in the tank is 6 g/liter? Circle your final answer (how many minutes until the salt concentration is 6 g/L).
\\\\
Let $b(t) =g$ of salt in the tank at time $t$, $t$ measured in minutes.  The concentration of salt in the tank at time $t$ is:
\begin{equation*}
\frac{b(t) g}{2000 L}
\end{equation*}
Then
\begin{align*}
\frac{db}{dt} &= \frac{.5 \text{g}}{\text{L}} \frac{10 \text{L}}{\text{min}} - \frac{b(t) \text{g}}{2,000 \text{\text{L}}}  \frac{10 L}{\text{min}}\\
                    &= \frac{5 \text{g}}{\text{min}} - \frac{b(t) \text{g}}{200 \text{ min}}\\
                    &= 5 - \frac{b}{200}
\end{align*}
IVP
\begin{align*}
b(0) &= \frac{35 g}{L} \cdot 2000 L \\
       &= 70,000g
\end{align*}
And
\begin{align*}
\frac{db}{dt} &= 5 - \frac{b}{200} \\
\frac{db}{5 - \frac{b}{200}} &= dt \\
\int \frac{db}{5 - \frac{b}{200}} &= \int dt \\
\frac{\text{ln} |5 - \frac{b}{200}| }{-\frac{1}{200}} &= t + c_1 \\
ln| 5 - \frac{b}{200}| &= \frac{t}{-200} + c_2 \\
e^{ln| 5 - \frac{b}{200}|} &= e^{\frac{t}{-200} + c_2 } \\
| 5 - \frac{b}{200}| &= e^{c_2}e^\frac{t}{-200}  \\ 
5 - \frac{b}{200} &= c_3 e^\frac{-t}{200}  \\
b &= c_4 e^{\frac{-t}{200}} + 1000
\end{align*}
General Solution
\begin{equation*}
b = c_4 e^{\frac{-t}{200}} + 1000
\end{equation*}
And
\begin{align*}
b(0) &= 70,000 \\
        &= c_4 e^0 + 1000 \\
        &= 69,000\\
        &= c_4
\end{align*}
Particular Solution
\begin{equation*}
b(t) = 69,000 e^{\frac{-t}{200}} + 1,000
\end{equation*}
When concentration is 6 g/L, 
\begin{align*}
\frac{b(t)}{2000} &= 6g/L\\
b(t) &= 12000 \\
12000 &= 69000e^{\frac{-t}{200}} + 1000 \\
\frac{11000}{69000} &= e^{\frac{-t}{200}} \\
\frac{11}{69} &= e^{\frac{-t}{200}} \\
\text{ln} \frac{11}{69} &= \text{ln}(e^{\frac{-t}{200}}) \\
-200 \cdot \text{ln} \frac{11}{69} &= t \\
t &= 367
\end{align*}
\end{problem}

\end{document}